%%%%%%%%%%%%%%%%%%%%%%%%%%%%%%%%%%%%%%%%%%%%%%%%%%%%%%%%%%%%%%%
%
% Welcome to Overleaf --- just edit your LaTeX on the left,
% and we'll compile it for you on the right. If you give
% someone the link to this page, they can edit at the same
% time. See the help menu above for more info. Enjoy!
%
%%%%%%%%%%%%%%%%%%%%%%%%%%%%%%%%%%%%%%%%%%%%%%%%%%%%%%%%%%%%%%%
% Preamble
% ---
\documentclass{article}

% Packages
% ---
\usepackage{amsmath} % Advanced math typesetting
\usepackage[utf8]{inputenc} % Unicode support (Umlauts etc.)
\usepackage[ngerman]{babel} % Change hyphenation rules
\usepackage{hyperref} % Add a link to your document
\usepackage{graphicx} % Add pictures to your document
\usepackage{listings} % Source code formatting and highlighting

%\usepackage[backend=bibtex,style=verbose-trad2]{biblatex} % Use biblatex package
%\bibliography{quick} % The name of the .bib file (name without .bib)

% Main document
% ---
\begin{document}
% Set up the maketitle command
\author{Claudio Vellage}
\title{A quick start to \LaTeX{}}
\date{\today{}} % You can remove \today{} and type a date manually

\maketitle{} % Generates title

\tableofcontents{} % Generates table of contents from sections

\section{How to use \LaTeX{}}
This is a quick introduction to the most common features of \LaTeX{}. For more features, check the other lessons on \url{http://www.latex-tutorial.com} and the articles on \url{http://blog.latex-tutorial.com}. 
\subsection{Document structure}
The documents in \LaTeX{} are structured slightly different from Word. The document consists of two parts, the preamble and the main document. If you're familiar with programming, then you might want to compare the structure with that of a C/C++ file.

\subsubsection{Preamble}
The purpose of the preamble is to tell \LaTeX{} what kind of document you will set up and what packages you are going to need. A package is a set of additional functions such as \emph{amsmath} for additional math formatting. For this document, the preamble looks like this:

\begin{lstlisting}[language={[LaTeX]TeX},caption=Preamble of this document,breaklines=true,frame=single]
% Preamble
% ---
\documentclass{article}

% Packages
% ---
\usepackage{amsmath} % Advanced math typesetting
\usepackage[utf8]{inputenc} % Unicode support (Umlauts etc.)
\usepackage[ngerman]{babel} % Change hyphenation rules
\usepackage{hyperref} % Add a link to your document
\usepackage{graphicx} % Add pictures to your document
\usepackage{listings} % Source code formatting and highlighting
\end{lstlisting}

You can set the class of the documentclass with the \emph{documentclass} command and add packages with the \emph{usepackage} command. Only the \emph{documentclass} command is mandatory, you can compile a document also without packages, yet some functions may be missing in this case. The \emph{usepackage} command \emph{must not} be used in the main document.

\subsubsection{Main document}
The main document is contained within the \emph{document} environment like this:

\begin{lstlisting}[language={[LaTeX]TeX},caption=Main part of a \LaTeX{} document.,breaklines=true,frame=single,frame=single]
\begin{document}
% ...
% ... Text goes here
% ...
\end{document}
\end{lstlisting}

Within those two statements, we can add the content of our document. But just adding the text is probably not enough, since we also have to apply formatting to it.

\subsection{Formatting in \LaTeX{}}
Formatting in \LaTeX{} can be applied by the use of commands and environment. The topmost environment is the document environment as described in Listing \ref{lst:main}. So there are obviously more environments, but how to find them? Well the easiest way is to download a \LaTeX{} cheat sheet which provides a list of the most useful commands and environments. For most packages there is also a manual available, which can be found on Google.

\subsubsection{Math typesetting}
To introduce you to math typesetting and environments, I will demonstrate you how to format some simple equations:

\begin{align}
f(x) &= x^2\\
f'(x) &= 2x\\
F(x) &= \int f(x)dx\\
F(x) &= \frac{1}{3}x^3
\end{align}

This can be done with the following code:

\begin{lstlisting}[language={[LaTeX]TeX},caption=Typesetting equations in \LaTeX{}.,breaklines=true,frame=single]
\begin{align}
f(x) &= x^2\\
f'(x) &= 2x\\
F(x) &= \int f(x)dx\\
F(x) &= \frac{1}{3}x^3
\end{align}
\end{lstlisting}

As you can see, we again have a \emph{begin} and \emph{end} statement, this it's applying the \emph{align} environment. This will align the equations at the ampersand (\&) sign. Naturally, those will be placed in front of the equality sign. If you watched carefully, you will see that \LaTeX{} magically added sequential numbers to all equations. \LaTeX{} does this for many other elements too. More about that in the next chapter.

\subsubsection{Document layout}
Usually a document does not only consist of a bunch of equations, but needs some kind of structure too. We are usually going to need at least:

\begin{itemize}
\item{Title/Titlepage}
\item{Table of contents}
\item{Headlines/Sections}
\item{Bibliography}
\end{itemize}

\LaTeX{} provides all the commands we need. The following commands will help us:\\

\begin{lstlisting}[language={[LaTeX]TeX},caption=Useful commands to structure a document.,label=lst:main,breaklines=true,frame=single]
\section{Text goes here} % On top of document hierarchy; automatically numbered
\subsection{}
\subsubsection{}
\paragraph{} % Paragraphs have no numbering
\subparagraph{}

\author{Claudio Vellage} % The authors name
\title{A quick start to \LaTeX{}} % The title of the document
\date{\today{}} % Sets date you can remove \today{} and type a date manually
\maketitle{} % Generates title
\tableofcontents{} % Generates table of contents from sections and subsections

\\ % Linebreak
\newpage{} % Pagebreak
\end{lstlisting}

There are commands to create sections, the sections are numbered automatically and the \emph{tableofcontents} command will use them to generate the table of contents. You don't have to do it yourself, ever. \LaTeX{} also provides commands to generate the title using the \emph{maketitle} command. This needs the \emph{author}, \emph{title} and \emph{date} command to be set. If you place the \emph{maketitle} or \emph{tableofcontents} command in your document, the Commands will be added at that exact place, so you probably want them in the very beginning of your document. If you want the title to appear on a single page, simply use the \emph{newline} command.

\subsubsection{Adding a picture}

Most documents will also need some kind of picture. Like this:

\begin{figure}
%\includegraphics{picture.png}
\caption{This figure shows the logo of my website.}
\end{figure}

 Adding them is fairly easy:

\begin{lstlisting}[language={[LaTeX]TeX},caption=Adding pictures in \LaTeX{}.,breaklines=true,frame=single]
\begin{figure}
\includegraphics[width=\textwidth]{picture.png}
\caption{This figure shows the logo of my website.}
\end{figure}
\end{lstlisting}

You have to embed the picture within the \emph{figure} environment, then use the \emph{includegraphics} command to select the image. Note that the picture file has to be in the same directory as your .tex file or you specify the path like this:
\begin{lstlisting}[language={[LaTeX]TeX},caption=How to specify a path.,breaklines=true,frame=single]
% ...
\includegraphics[width=\textwidth]{FOLDERNAME/picture.png}
% ...
\end{lstlisting}

It makes sense to manage all your pictures in subfolders if you have many of them.

\subsection{Sourcecode formatting}

Throughout the book I used the \emph{listings} package to format the \LaTeX{} code snippets. You can specify the language for each \emph{lstlistings} block, in this case i used \LaTeX{} of course and added a caption as well as enabled linebreaking:

\begin{lstlisting}[language={[LaTeX]TeX},caption=How to use the listings package.",breaklines=true,frame=single,escapechar=^]
^\textbackslash^begin{lstlisting}[language={[LaTeX]TeX},caption=,breaklines=true,frame=single]
% Source code goes here
^\textbackslash^end{lstlisting}
\end{lstlisting}

The language is specified as \emph{[DIALECT]LANGUAGE}, here \LaTeX{} is dialect of \TeX{}. A list of all programming languages can be obtained from the \emph{listings} package manual, but you can also just try out if it works.

\subsubsection{References and Bibliography}\label{sec:ref}

You can specify labels for all the things that are automatically numbered. If you want to refer to a section of your document, you'd simply use the \emph{label} and \emph{ref} (reference) command. Where the label indicates what you want to refer to and the reference will print the actual number of the element in your document. This will also work interactively in your PDF reader You can try this feature in section \ref{sec:ref}.

\begin{lstlisting}[language={[LaTeX]TeX},caption=Labels and references in \LaTeX{},breaklines=true,frame=single]
\section{}\label{sec:YOURLABEL}
% ...
I've written text in section \ref{sec:YOURLABEL}.
\end{lstlisting}

Papers usually include a lot of references to the great works of other people. In order to properly cite them, we'd want to use the biblatex package. For this purpose we'd simply add the following code to our preamble:

\begin{lstlisting}[language={[LaTeX]TeX},caption=Preamble code to use biblatex.,breaklines=true,frame=single]
\usepackage[backend=bibtex,style=verbose-trad2]{biblatex} % Use biblatex package
\bibliography{FILENAME} % The name of the .bib file (name without .bib)
\end{lstlisting}

All bibliographic information will be stored in the bibliography (.bib) and \emph{must not} be inside of the .tex file An example could look like this:

\begin{lstlisting}[language={[LaTeX]TeX},caption=Example bibliography file.,breaklines=true,frame=single]
@ARTICLE=
{
VELLAGE:1,
AUTHOR="Claudio Vellage",
TITLE="A quick start to \LaTeX{}",
YEAR="2013",
PUBLISHER="",
}
\end{lstlisting}

Now i could add a self reference using the \emph{cite} command:

\begin{lstlisting}[language={[LaTeX]TeX},caption=The cite command.,breaklines=true,frame=single]
This feature works as I described in \cite{VELLAGE:1}.
\end{lstlisting}

The \emph{biblatex} is very smart and wil print autogenerate the bibliography if we want to. We'd usually do this in the end of the document. Simply add

\begin{lstlisting}[language={[LaTeX]TeX},caption=The cite command.,breaklines=true,frame=single]
\printbibliography
\end{lstlisting}

to our document. More examples can be found on the \href{http://www.latex-tutorial.com/lesson7/}{website}.

\end{document}